\documentclass[12pt]{extarticle}
\usepackage[utf8]{inputenc}
\usepackage{hyperref}
\usepackage{apacite}

\title{Adversarial Attack on Image Annotation}
\author{H.J.M. van Genuchten, 1297333}
\date{March 2020}

\begin{document}

\maketitle

\section{ Goal}
\subsection{General}
Goal of the BEP is to generate a new benchmark tool for Natural Language Processing (NLP). Preferably one that is adversarial as it is less susceptible to overfitting, and can find weaknesses in current state of the art methods.

\subsection{Specific}
I want to create an adversarial attack benchmark targeted at image annotation \cite{venkatesh,xu2016show}. Inspired by "Adversarial Examples" proposed by \citeauthor{szegedy2014intriguing}, which are images with small perturbations that is able to throw off classifcation models, but are unnoticable to humans. \citeauthor{szegedy2014intriguing} also found that these adversarial examples generalize across models. The research questions then boil down to:
\begin{itemize}
    \item Are Image Annotation Networks vulnerable to adversarial examples?
    \item Is the output annotation controlable? (i.e. can we produce an adversarial example which will generate a chosen output)
\end{itemize}


I want to definetly answer the first research question. I will try my best to answer the second.

To minimize the initial scope, I will focus on a single image annotation model, most likely Show, Attend and Tell (S.A.T.) by \citeauthor{xu2016show}. Although this project will focus on Adversarial Attack on S.A.T., it should be applicable to the broader scope of image annotation and image-to-text models. However I do not plan to go into that.

% \subsection{Personal}
% Although not directly related, I have some personal goals that I would like to mention and rationalize:
% \begin{itemize}
%     \item  Test Driven Development (TDD): To ensure that the code that is written is correct, modular and easy to change. Also making sure everything is deterministic and therefore reproducable.
%     \item  A better understanding of how Neural Networks work and train. NNs are usefull but to most still a magic black box that just works. I want to be able to better reason as to why certain things will work and others won't.
%     \item  Mono-repo: Keeping everything related to the project under a single github repository. Including the code, paper and other resources.
% \end{itemize}

\section{Pre Study}
Relevant research has been done in the area of image classification, most notably by \citeauthor{goodfellow2015explaining}, who propose a faster way of finding adversarial examples for image classification. Also the findings by \citeauthor{venkatesh} show that adversarial examples have a cross-model generalization. The result of this project could thus also be used to strengthen current datasets. Furthermore there has been a lot of research into image annotation. One of the more basic full deep learning image annotation models, S.A.T. \cite{xu2016show}, will be my first focus, as it is closly related to the image classification structure used in aformentioned research.

\section{Methodology}
First reproducing some relevant papers as to gain experience with the field and project. Starting with (re)producing an adversarial attack on an mnist classification model \cite{szegedy2014intriguing} and then applying that on an a deep learning image annotation model, like the one proposed by \citeauthor{xu2016show}. After which I will combine the two methods to see if S.A.T. is susceptible to adversarial examples. For training and testing I will make use of the publicly available datasets MS COCO \cite{lin2015microsoft} and Flickr8K \cite{Flickr8k}. I will be mainly looking at BLUE \cite{papineni_roukos_ward_zhu_2001} score.

\subsection{Timeline}
I have set up the following schedule for myself. Bolded deadlines are from the university, the rest is a rough sketch to keep myself on schedule.
\begin{table}[h]
    \begin{tabular}{|l|l|}
        \hline
        \multicolumn{1}{|r|}{Date} & Description                                           \\ \hline
        22 February                & Hand in draft of Project plan                         \\ \hline
        \textbf{01 March}          & \textbf{Hand in Project plan}                         \\ \hline
        08 March                   & Reproduced Adversarial Attack on mnist classification \\ \hline
        22 March                   & Have a working Image Annotation model                 \\ \hline
        10 April                   & Applied Adversarial Attack on Image Annotation        \\ \hline
        \textbf{17 April}          & \textbf{Hand in Partial thesis}                       \\ \hline
        06 May                     & Targeted Output                                       \\ \hline
        03 June                    & Finalized experimentation                             \\ \hline
        \textbf{19 June}           & \textbf{Hand in Final Thesis}                         \\ \hline
    \end{tabular}
\end{table}


\subsection{Technology}
The code will be written in Python, with Pytorch as Machine-Learning backend.
Versioning will be done with git.
The repo can be found on \href{https://github.com/dikvangenuchten/bep-adversarial-image-annotation}{my personal github repository}. Which will also contain the working version of the paper and other resources, such as this plan.


\bibliographystyle{apacite}
\bibliography{project_plan}

\end{document}
