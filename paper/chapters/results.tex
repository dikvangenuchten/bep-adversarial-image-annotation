* Currently all plots are based on a subset of 120 images, my intention is to rerun them on the complete test set (5k images), which depending on the amount iterations this takes about 2 hours per epsilon at 100 iterations.

\subsection*{Adversarial Samples}



\subsection*{Distracting Samples}
To distract the model, adversarial samples are created using the iterative method (EQ. \ref{IterativeMethod}) and the distraction adversarial loss (EQ. \ref{distraction_loss}). The amount of iterations was experimentally found to be good enough in most cases at 100, in which more would result in better distraction at the cost of longer running times. The top left pixel was chosen to focus the attention on as the model focus least on it (\ref{average_attention_clean}) (albeit slightly) during the clean images. With an epsilon of 0.04 satisfactory results are achieved. The attention of the model clearly focused on the top left on average as can be seen in figure \ref{average_attention_adv}. With the perturbation at most 0.04 the image is visually almost identical to the human eye \ref{adv_example_img}.

\begin{figure*}
    \centering
    \begin{minipage}{0.45\textwidth}
        \centering
        \includegraphics[width=0.9\textwidth]{figures/distraction_attention_epsilon_0.png} % first figure itself
        \caption{Average attention on clean images}
        \label{average_attention_clean}
    \end{minipage}\hfill
    \begin{minipage}{0.45\textwidth}
        \centering
        \includegraphics[width=0.9\textwidth]{figures/distraction_attention_epsilon_0.04.png} % second figure itself
        \caption{Average attention on adversarial images with $\epsilon$=0.04 at 100 iterations}
        \label{average_attention_adv}
    \end{minipage}
\end{figure*}

\begin{figure*}
    \centering
    \begin{minipage}{0.45\textwidth}
        \centering
        \includegraphics[width=0.9\textwidth]{figures/distraction_adv_sample_0.png} % first figure itself
    \end{minipage}\hfill
    \begin{minipage}{0.45\textwidth}
        \centering
        \includegraphics[width=0.9\textwidth]{figures/distraction_adv_sample_0.04.png} % second figure itself
    \end{minipage}
    \caption{Clean Image (left), Adversarial Image $\epsilon=0.04, N=100$ (right)}
    \label{adv_example_img}
\end{figure*}

The attention and sentences for the figure \ref{adv_example_img} are visualized in figure \ref{adv_example_att}. The model still attends to other parts of the image, however they are not clearly a single object relating to the word that is generated. During the generations of the last few words the attention is focused almost solely on the top left part. 

\begin{figure*}
    \centering
    \begin{minipage}{0.45\textwidth}
        \centering
        \includegraphics[width=0.9\textwidth]{figures/caption_distraction_adv_sample_0.png} % first figure itself
    \end{minipage}\hfill
    \begin{minipage}{0.45\textwidth}
        \centering
        \includegraphics[width=0.9\textwidth]{figures/caption_distraction_adv_sample_0.04.png} % second figure itself
    \end{minipage}
    \caption{Attention on Clean Image (left) and Adversarial Image $\epsilon=0.04, N=100$ (right)}
    \label{adv_example_att}
\end{figure*}


\begin{figure}
    \includegraphics[width=0.9\textwidth]{figures/distraction_bleu_score_over_epsilon.png}
    \caption{BLEU score during distraction over epsilon}
\end{figure}



% Although I currently don't have complete results. I do have some initial samples that worked. I am still in the process of calculating the BLEU score and cosine similarity over the whole datasets.
% Preliminary results images:

\begin{figure*}
    \centering
    \begin{minipage}{0.45\textwidth}
        \centering
        \includegraphics[width=0.9\textwidth]{figures/caption_teddy_normal.png} % first figure itself
        \caption{Prediction by Show Attend and Tell on a normal image}
    \end{minipage}\hfill
    \begin{minipage}{0.45\textwidth}
        \centering
        \includegraphics[width=0.9\textwidth]{figures/caption_teddy_adversarial_0.2.png} % second figure itself
        \caption{Prediction on an adversarial image with $\epsilon=0.2$ (roughly $5\%$ of original range)}
    \end{minipage}
\end{figure*}

\begin{figure*}
    \centering
    \begin{minipage}{0.45\textwidth}
        \centering
        \includegraphics[width=0.9\textwidth]{figures/caption_baseball_normal.png} % first figure itself
        \caption{Prediction by Show Attend and Tell on a normal image}
    \end{minipage}\hfill
    \begin{minipage}{0.45\textwidth}
        \centering
        \includegraphics[width=0.9\textwidth]{figures/caption_baseball_adversarial.png} % second figure itself
        \caption{Prediction on an adversarial image with $\epsilon=0.2$ (roughly $5\%$ of original range)}
    \end{minipage}
\end{figure*}

\begin{figure*}
    \centering
    \begin{minipage}{0.45\textwidth}
        \centering
        \includegraphics[width=0.9\textwidth]{figures/caption_kitchen_normal.png} % first figure itself
        \caption{Prediction by Show Attend and Tell on a normal image}
    \end{minipage}\hfill
    \begin{minipage}{0.45\textwidth}
        \centering
        \includegraphics[width=0.9\textwidth]{figures/caption_kitchen_adversarial.png} % second figure itself
        \caption{Prediction on an adversarial image with $\epsilon=0.2$ (roughly $5\%$ of original range)}
    \end{minipage}
\end{figure*}

\begin{figure*}
    \centering
    \begin{minipage}{0.45\textwidth}
        \centering
        \includegraphics[width=0.9\textwidth]{figures/caption_elephant_normal.png} % first figure itself
        \caption{Prediction by Show Attend and Tell on a normal image}
    \end{minipage}\hfill
    \begin{minipage}{0.45\textwidth}
        \centering
        \includegraphics[width=0.9\textwidth]{figures/caption_elephant_adversarial.png} % second figure itself
        \caption{Prediction on an adversarial image with $\epsilon=0.2$ (roughly $5\%$ of original range)}
    \end{minipage}
\end{figure*}